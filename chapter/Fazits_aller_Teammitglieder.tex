%!TEX root = ../Thesis.tex
\section{Fazits aller Teammitglieder}

\subsection{Dennis Gentges}
Ich empfand es als sehr gut, dass das Modul WIP nicht ausschließlich die reine Programmierung einer Anwendung beinhaltete, sondern ebenfalls das Projektmanagement als essenzieller Bestandteil für eine erfolgreiche Implementierung benötigt wurde. Verglichen mit bereits bekannten Projekten aus dem Betrieb finde ich dies realitätsnaher als es in anderen Modulen der Fall ist. Ein Projekt vom Anfang bis zum Ende zusammen mit einem Team durchzuführen hat mir persönlich viele neue Erkenntnisse und Erfahrungen gebracht.

Neben grundlegenden Projektmanagementmethoden wurden während des Projektes allerdings auch andere Fähigkeiten benötigt, welche ich zu Beginn des Projektes noch nicht beherrschte. Dazu gehörte unter anderem die Arbeit mit GitHub. Dabei benötigte ich eine wesentlich höhere Einarbeitungszeit als andere Kollegen da ich hierbei noch keine Erfahrungen gesammelt hatte. Nach dieser Einarbeitungszeit und der Unterstützung der erfahreneren Teammitglieder verstand ich jedoch die Grundfunktionen, sodass ich diese Fähigkeiten nun auch in meinem weiteren Arbeits- und Studienleben einsetzen kann. 
Neben GitHub war mir auch das Textverarbeitungsprogramm Latex noch fremd. Da uns hier ebenfalls jegliche Einweisung fehlte, benötigte ich hierbei ebenfalls wieder eine längere Einarbeitungsphase.

In unserem Projektteam gab es einige sehr gute und geübte Programmierer und ich empfand es als eine große Herausforderung mit dieser Leistung mithalten zu können. Da ich zuvor keine größeren Erfahrungen im Programmieren gesammelt habe und dies auf der Arbeit auch nur begrenzt tun muss, benötigte ich einen wesentlich höheren Zeitaufwand mich in umgebenden Klassen und Methoden einzuarbeiten.

Da meine eigentliche zugewiesene Hauptaufgabe in der Projektsteuerung lag, konzentrierte ich mich nun zunächst darauf. Unter der Betrachtung der Meilensteine wurden diese alle fristgerecht erfüllt bzw. eingehalten und wir kamen dem Projektziel Schritt für Schritt näher.

Zusätzlich habe ich den Zeitaufwand für die Fertigstellung der Ausarbeitung unterschätzt. Insbesondere im Zeitraum zwischen November und Februar fielen einige Prüfungsleistungen in der FHDW, aber auch der Abschluss unserer Ausbildung zum Fachinformatiker an. Dies war für alle Teammitglieder eine ziemlich große, zusätzliche Belastung. Am Ende hat es aber dennoch funktioniert.

\subsection{Tom Bockhorn}
Das Projekt zur Erstellung eines Taschenrechners in polnischer Notation sehe ich bereits vor der Abgabe als erfolgreich an. Technisch gesehen sind wir den funktionalen Anforderungen nachgekommen, konnten das Projekt innerhalb des zeitlichen Rahmens zu einem akzeptablen Stand bringen und haben kreative Lösungen für die offenen Fragen der Aufgabenstellung gefunden. Im Laufe des Projektes bekam ich die Gelegenheit mir auf eigene Faust, aber auch in Kommunikation mit meinem Team neue, spannende Konzepte der Projektdurchführung und Programmierung anzueignen. Auch das zuvor angebaute Wissen aus dem Studium konnte ich einigermaßen in das Projekt einbringen, sodass diese Inhalte nur vertieft wurden. Allein die Größe des Teams bestärkte die Relevanz von Organisation und Kommunikation untereinander. Ich finde die Aufteilung in ein Frontend und Backend Team hat sehr gut funktioniert und selbst zu kritischen Zeitpunkten konnten wir auf Verstärkung durch die anderen zählen.

Für mich persönlich war die Einarbeitung in die Programmierung nur ein geringes Problem, da ich bereits viel Erfahrung mit Java und ein wenig Erfahrung mit Android gewinnen durfte. Dafür stellte sich die Umsetzung der Ausarbeitung in Latex als größere Herausforderung heraus. Dort möchte ich aber unser Team loben, denn es hat sich immer jemand gefunden der Zeit und Motivation hatte sich in uns unbekannte Themen einzuarbeiten!


\subsection{Hendrik Falk}
Aus meiner Sicht war die Projektdurchführung erfolgreich. Die Anforderungen vom Auftraggeber konnten erfolgreich und fristgerecht umgesetzt werden. Dazu zählt neben der Applikation selbst auch die Dokumentation. 

Meine dispositiven Aufgaben stellten eine sehr neue Aufgabe für mich dar und es bereitete mir Freude daran zu Arbeiten. Sowohl der Umgang mit Android Studio, als auch mich Latex war mir neu und stellte deshalb eine große Herausforderung dar, welche ich aber im für das Projekt benötigten Rahmen bewältigen konnte. Nicht zuletzt dadurch, dass gute Vorarbeit geleistet wurde und How-Tos erstellt wurden, die den Einstieg erleichterten. Die größte Herausforderung war die Überlappung des Projektes mit diversen Prüfungsleistungen im Studium, sowie den Abschlussprüfung unserer Ausbildung als Fachinformatiker. Letzteres fiel genau in die Zeit vor Projektabschluss und nahm viele unserer Ressourcen in Anspruch. 

Dazu kam noch, dass das Team auf verschiedene Standorte verteilt war. Darunter auch Wuppertal und Monheim. Außerdem hatten wir einen Kollegen, der von einer anderen Firma beschäftigt war. Dadurch musste viel Kommunikation digital stattfinden, was Selbige erschwerte. 

Die Kommunikation innerhalb des Teams habe ich als sehr positiv wahrgenommen und war immer sehr problemorientiert. Auch in schwierigen Situationen konnten wir Zusammenhalt beweisen und weiterkommen und somit das Projekt erfolgreich abschließen, wobei ich viele neue Erfahrungen machte, die mir auch im späteren Berufsleben weiterhelfen werden können.

\subsection{Getuart Istogu}
Rückblickend konnte ich aus dem Projekt viele neue Erkenntnisse gewinnen. Zumal war es mein erstes Projekt mit mehreren Beteiligten. Dadurch wurde mir bewusst, wie wichtig eine gute Aufgabenteilung und Absprache innerhalb eines Teams ist. Außerdem wurde mir durch das Projekt bewusst, wie wichtig eine gute Dokumentation ist, da ich für meine Entwicklung nachvollziehen können musste, was das Frontend gemacht hat, um bei der Implementierung der Menüführung auszuhelfen. Zudem war die Dokumentation ein wichtiger Bestandteil bei der Wiederaufnahme der Programmierung nach einer längeren Pause, bedingt durch weitere Prüfungsleistungen im Studium und der Endphase des IHK-Abschlusses als Fachinformatiker.

Generell verlief das Projekt aus meiner Sicht gut. Wir konnten die Anforderungen fristgerecht erfüllen. Dabei konnte ich vor allem den Umgang von GitHub und Latex erlernen. Ich lernte Latex Wert zu schätzen, dadurch dass die Ausarbeitung ohne viel Mehraufwand sehr einheitlich aussah. Zuerst viel mir der Umgang damit jedoch schwer. Die Einarbeitung in GitHub war für mich auch sehr aufwändig, jedoch konnte ich von dem Wissen der anderen Teammitglieder profitieren. 

Des Weiteren hatte ich bereits eine mobile App für iOS entwickelt. Dafür habe ich aber die Programmiersprache C\# und das dazugehörige Framework Xamarin verwendet. Was mir dabei aufgefallen ist, dass besonders im Frontendbereich sehr viele Ähnlichkeiten gab und das Einarbeiten vereinfacht hat. Weitere auffallende Gemeinsamkeiten beziehungsweise Unterschiede sind mir nicht aufgefallen. 

Für das Backend habe ich erstmal Unittests in einem Projekt implementiert und war erstaunt, wie einfach sie zu implementieren sind. Besonders der Aspekt, dass leicht nachvollzogen werden kann, ob die Methoden wie erwartet funktionieren. Dies kann dann auch losgelöst vom Frontend überprüft werden, dass auch wichtig für das Projekt waren.
Ich hatte am Anfang meine Bedenken gehabt, ob wir die Anforderungen an den Taschenrechner umsetzen konnten, aber diese Bedenken lösten sich im Laufe des Projektes auf.


\subsection{Jannis Keienburg}
Für mich war das das erste Programmierprojekt mit mehreren Beteiligten. Ebenfalls war es die erste große Java- bzw. Android-Applikation, an der ich mitentwickelt habe. Die größte Herausforderung aus meiner Sicht war die Kommunikation innerhalb des Teams. Es war teilweise schwierig zu erfahren, wer gerade bei welchem Thema wie weit war. Innerhalb meines Aufgabenbereiches war die größte Herausforderung für mich die Programmierung der Klasse zur Berechnung der Hoch- und Tiefpunkte. Grund dafür war zum einen die Abhängigkeit zu der Klasse Zeros, aber auch viele kleine Fehler, die erst mit zeilenweisem Debugging gefunden und gelöst werden konnten. Das Lesen der Codebeiträge der anderen Beteiligten war anfangs etwas schwierig, im Laufe des Projektverlaufes wurde dies allerdings immer einfacher, auch weil die Klassen gut kommentiert wurden. Alles in allem ein lehrreiches und für den späteren Berufsweg lehrreiches Projekt.

\subsection{Tim Jonas Meinerzhagen}
Ich werde nachfolgend ein kurzes Fazit zur erstellten App, unserer Arbeit im Team und meinem Beitrag zum Projekt geben.

Die von uns erstellte App konnte allen gestellten Anforderungen entsprechend umgesetzt werden. Es wurde eine breite Spanne an Funktionalitäten implementiert, welche verschiedenste Bereiche abdeckt. Alle Kacheln können dynamisch angepasst werden. In der Ausführung ist die App sehr stabil und es treten keine größeren Bugs auf, die einen Crash herbeiführen könnten. An einigen Stellen wären noch Anpassungen hilfreich gewesen, um eine bessere User Experience herbeizuführen, dies wurde allerdings auf Grund der begrenzten Zeit als akzeptabel angesehen. Insgesamt bin ich sehr zufrieden mit dem letztlichen Stand der App.

Das Projekt in unserem Team „Das Proletariat“ ist sehr harmonisch verlaufen. Alle Teammitglieder konnten ihren Beitrag zum Projekterfolg leisten. Es war das erste Mal, dass wir ein Softwareprojekt unter Beteiligung von acht Entwicklern durchführen mussten. Dies erforderte eine umfassende Organisation, welche Dennis und Hendrik herausragend geleistet konnten. Dadurch, dass wir auch bereits seit mehreren Jahren zusammen studieren war die Zusammenarbeit sehr harmonisch, was auch bei Meinungsverschiedenheiten beibehalten werden konnte.  Die Projektdurchführung im Team habe ich als sehr produktiv war genommen.

Als Teil des Architektur-Teams, welches eine Schnittstellenfunktion darstellte, war ich an einem breiten Spektrum von Aufgaben beteiligt. Dies hat dazu geführt, dass ich einen umfassenden Einblick in die verschiedenen Strukturen des Projektes erhalten konnte. 
Letztlich komme ich noch zu meinem eigenen Beitrag zum Projekt. Dadurch, dass ich bereits Vorerfahrung mit der Programmierung von Android Apps hatte, konnte ich während der Planung und der Umsetzung als Experte in diesem Bereich fungieren. So konnte ich mich schneller in die verschiedenen Bereiche der App einarbeiten und helfen, wenn notwendig. 


\subsection{Khang Pham}
Insgesamt kann die Durchführung des Projekts als erfolgreich angesehen werden. Die Anforderungen wie sie in der Aufgabenbeschreibung beschrieben sind und wie sie mit dem Auftraggeber besprochen wurden, konnten allesamt umgesetzt bzw. in Ansätzen umgesetzt werden (Das heißt es konnten alle Operatoren umgesetzt werden. Der Operand ‘‘Multiplikation’‘ beispielsweise, ist allerdings für den Operator ‘‘Matrizen’‘ nicht funktional). Da die Taschenrechner-Applikation jedoch nicht für den tatsächlichen Einsatz mit echten Anwendern entwickelt wurde, war eine hundertprozentige Umsetzung aller Funktionen, auch in Anbetracht des zeitlichen Rahmens kein Ziel des Projektes. Vielmehr konnte eine Applikation entwickelt werden, welche alle Grundanforderungen eines Taschenrechners mit UPN umsetzt und zusätzlich das innovative Konzept der Gruppe für ein Kachellayout und für Usability demonstriert. Die Recherche, die das Team außerhalb der Vorlesungen durchführen musste um die Entwicklung einer App in Android zu erlernen, das Koordinieren zwischen den einzelnen Teammitgliedern für die Durchführung und Zeitplanung des Projektes sowie die Verteilung der Arbeitspakete innerhalb des Projektes bildete für die Projektteilnehmer eine neue, wertvolle Erfahrung.
Insbesondere für mich war die Arbeit an einem Softwareprojekt als einer von vielen Entwicklern eine neuartige und sehr interessante Erfahrung. Auch die Arbeit mit GIT empfand ich als sehr interessant, da ich auf der Arbeit noch nicht damit arbeiten konnte. Der benötigte Verwaltungsoverhead für ein solches Projekt mit einer großen Anzahl an beteiligten Entwicklern stellte sich als deutlich größer heraus, als ich ursprünglich annahm. Insgesamt sammelte ich während des Projektes viele neue und wertvolle Erfahrungen für mich, die auch für meine zukünftige Arbeit als IT-Consultant relevant werden könnten. 

\subsection{Tim Schwenke}
Was hab ich gelernt, was lief gut, was lief nicht gut.
XXX

\subsection{Team}
Insgesamt kann die Durchführung des Projekts als erfolgreich angesehen werden. Die Anforderungen wie sie in der Aufgabenbeschreibung beschrieben sind und wie sie mit dem Auftraggeber besprochen wurden, konnten allesamt umgesetzt bzw. in Ansätzen umgesetzt werden (Das heißt es konnten alle Operatoren umgesetzt werden. Der Operator ‘‘Multiplikation’‘ beispielsweise, ist allerdings für den Operand ‘‘Matrizen’‘ nicht funktional). Es konnte eine Applikation entwickelt werden, welche alle Grundanforderungen eines Taschenrechners mit UPN umsetzt und zusätzlich das Konzept für ein Kachellayout und für Usability demonstriert. Hierbei sollte auch die vorbildliche Zusammenarbeit der Teammitglieder während des gesamten Zeitraums des Projektes hervorgehoben werden. Alle Herausforderungen, die die Durchführung eines solchen Projektes mit sich zogen, konnten von den Teammitgliedern in Zusammenarbeit gemeistert werden. Dabei hatte jeder Projektteilnehmer eigene Problembereiche und Schwierigkeiten im Rahmen des Projektes, dennoch waren die Teammitglieder jederzeit dazu bereit einander auszuhelfen oder neue Dinge beizubringen. Die Erfahrungen, die das Team während des Moduls gesammelt hat, können insgesamt also nur als positiv angesehen werden. 
