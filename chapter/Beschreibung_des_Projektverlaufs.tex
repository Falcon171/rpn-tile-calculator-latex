%!TEX root = ../Thesis.tex
\section{Beschreibung des Projektverlaufs}

\subsection{Tatsächliche Aufgabenverteilung im Team}

XXX

\clearpage

\subsection{Teammeeting-Protokolle}

{\def\arraystretch{1.25}\tabcolsep=5pt
\begin{longtable}{|l|l|p{11cm}|}
	\hline
	\textbf{Datum} & \textbf{Dauer} & \textbf{Beschreibung}
	\\ \hline \hline
	\endfirsthead
	
	\hline
	\endhead
	
	\hline
	\endfoot
	
	\multicolumn{3}{|c|}{\textit{Summe der Dauer aller Gruppenmeetings: 1.180 Minuten}}
	\\ \hline\hline
	\multicolumn{3}{|c|}{\textit{Summe der Dauer aller GUI-Team-Meetings: 1.020 Minuten}}
	\\ \hline\hline
	\multicolumn{3}{|c|}{\textit{Summe der Dauer aller Architektur-Team-Meetings: 360 Minuten}}
	\\ \hline\hline
	\multicolumn{3}{|c|}{\textit{Summe der Dauer aller Backend-Meetings: 110 Minuten}}
	\\ \hline
	\endlastfoot
	
		\textbf{03.09.2019} 
			& \hfill\textbf{200} & \textbf{Gruppenmeeting 1 - Projektauswahl}
			\\ & &
			\small{\textit{Teilnehmer: Bockhorn, Falk, Gentges, Istogu, Keienburg, Meinerzhagen, Pham, Schwenke}}
			\\ & &
			Auswahl des Projekttyps. Entscheidung für die Entwicklung einer Android-App.
			\\ & &
			Erste Einarbeitung in die Thematik. Lesen des bereitgestellten Dokuments mit Aufgabenstellung, groben Anforderungen und organisatorisches. 
			\\ & &
			Konzepterarbeitung auf Papier. Vorstellung und Diskussion verschiedener Ansätze.
	\\\hline
		\textbf{04.09.2019} 
			& \textbf{\hfill110} & \textbf{Gruppenmeeting 2 - Grobes Konzept}
			\\ & &
			\small{\textit{Teilnehmer: Bockhorn, Falk, Gentges, Istogu, Keienburg, Meinerzhagen, Pham, Schwenke, Herr Prof. Dr. Seifert}}
			\\ & &
			Einigung mit dem Dozenten auf einen Ansatz für den Taschenrechner.
			\\ & &
			Absprachen über Meeting-Protokolle, Studienarbeit und Projekttagebücher
			\\ & &
			Ausarbeitung des Konzepts für den Taschenrechner. Hier wurden dem Auftraggeber Herr Seifert mehrere Konzepte vorgestellt und gemeinsam mit ihm genaue Anforderungen erarbeitet.
			\\ & &
			Aufteilung der Gruppe in das GUI-, Architektur- und Backend Team
	\\\hline
		\textbf{05.09.2019} 
			& \textbf{\hfill60} & \textbf{GUI-Team-Meeting 1 - Prototyp} 
			\\ & &
			\small{\textit{Teilnehmer: Bockhorn, Gentges, Pham}}
			\\ & &
			Erstellung des Mid-Fidelity Prototypen in Power Point.			
		\\ \cline{2-3}
		& \textbf{\hfill60} & \textbf{Architektur-Team-Meeting 1 - Konzept} 
			\\ & &
			\small{\textit{Teilnehmer: Falk, Meinerzhagen, Schwenke}}
			\\ & &
			Grobe Konzeptionierung der Architektur des Projektes inklusive grobem Aufbau des Projektes und der Schnittstellen  
		\\ \cline{2-3}
		& \textbf{\hfill60} & \textbf{Backend-Team-Meeting 1 - Funktionen sammeln} 
			\\ & &
			\small{\textit{Teilnehmer: Istogu, Keienburg}}
			\\ & &
			Brainstorming über geplante Operatoren, Operanden und Einstellungen
		\\ \cline{2-3}
		& \textbf{\hfill60} & \textbf{Gruppenmeeting 3 - Abstimmung der Teams} 
			\\ & &
			\small{\textit{Teilnehmer: Bockhorn, Falk, Gentges, Istogu, Keienburg, Meinerzhagen, Pham, Schwenke}}
			\\ & &
			Besprechung des aktuellen Stands der Architektur, der Backend-Ideen und des Prototypen.
			\\ & &
			Diskussion über Umsetzung und Workflow der App.
			\begin{itemize}\renewcommand\labelitemi{--}
				\item  Wie sollen die einzelnen Kacheln funktionieren?
				\item Wie sollen die Kacheln miteinander interagieren?
				\item Wie könnte die Architektur der App aussehen?
			\end{itemize}
			\\ & &
			Absprache der Projektplanungsergebnisse
	\\\hline
		\textbf{13.09.2019} 
			& \hfill\textbf{120} & \textbf{GUI-Team-Meeting 2 - Mid-Fidelity Prototyp} 
			\\ & &
			\small{\textit{Teilnehmer: Bockhorn, Gentges, Pham}}
			\\ & &
			Erweiterung des Mid-Fidelity Prototyps in PowerPoint inlusive Workflow mit Links
			\\ & &
			Finalisierung der geplanten Funktionalitäten des Prototypen
	\\\hline
		\textbf{17.09.2019} 
			& \hfill \textbf{ 60} & \textbf{Architektur-Team-Meeting 2 - Klassendiagramm}
			\\ & &
			\small{\textit{Teilnehmer: Falk, Istogu, Meinerzhagen, Schwenke}}
			\\ & &
			Erweiterung UML-Klassendiagramm. Die Klasse Operand wird abstrakt und wird von konkreten Operanden wie Vector geerbt. Diese stellen Extensions dar die neben den eigentlichen mathematischen Werten weitere Daten und Verhalten mitbringen.
			\\ & &
			Welche Library soll für Mathe-Funktionalitäten benutzt werden? JScience und die bereits mitgelieferte Standardbibliothek.
			\\ & &
			Darstellungsformen der Elemente mit ASCI-Art oder LaTeX
		\\ \cline{2-3}
		& \textbf{\hfill90} & \textbf{GUI-Team-Meeting 3 - Workflow} 
			\\ & &
			\small{\textit{Teilnehmer: Bockhorn, Gentges, Pham}}
			\\ & &
			Wie sollen Elemente in der GUI dargestellt werden? Als Character oder gerendert in LaTeX. Letzteres ist mit höherer Komplexität verbunden sieht aber auch besser aus. 
			\\ & &
			Wie soll das Layout funktionieren? Gridlayout fällt raus, weil nicht dynamisch genug? Relative-Layout ist eine Option. Hier darf aber die Anordnung beim Rotieren nicht unkontrolliert verändert werden. UI Team möchte, dass alle Komponenten gleich groß sind. In dem Fall kann man Gridlayout benutzen.
			\\ & &
			Wie soll die Eingabe von Funktionen im Graph Operand funktionieren? Nur möglich mit bereits vorhandenen Elementen in der Oberfläche. Es öffnet sich keine Tastatur.
	
	\\\hline
		\textbf{23.09.2019} 
			& \textbf{\hfill120} & \textbf{GUI-Team-Meeting 4 - Prototyp} 
			\\ & &
			\small{\textit{Teilnehmer: Bockhorn, Gentges, Pham}}
			\\ & &
			Finalisierung des Mid-Fidelity Prototyps inklusive der Funktionalitäten
			\\ & &
			Durchplanung des Workflows			
	\\\hline
		\textbf{26.09.2019} 
			& \textbf{\hfill150} & \textbf{Gruppenmeeting 4 - Abstimmung der Teams}
			\\ & &
			\small{\textit{Teilnehmer: Bockhorn, Falk, Gentges, Istogu, Keienburg, Meinerzhagen, Pham, Schwenke}}
			\\ & &
			Vorstellung des Mid-Fidelity Prototyps und Einholung von Feedback
			\\ & &
			Erklärung der geplanten Systemarchitektur
			\\ & &
			Darstellung der geplanten Programmodule			
	\\\hline
		\textbf{09.10.2019} 
			& \textbf{\hfill90} & \textbf{Gruppenmeeting 5 - Abstimmung der Teams}
			\\ & &
			\small{\textit{Teilnehmer: Bockhorn, Falk, Gentges, Istogu, Keienburg, Meinerzhagen, Pham, Schwenke}}
			\\ & &
			Vorstellung des Backend-Entwurfs für Teammitglieder, die für das Frontend zuständig sind.
			\\ & &
			Vorstellung des Frontend-Entwurfs für Teammitglieder, die für das Backend zuständig sind.
			\\ & &
			Diskussion über Verbindung von Frontend und Backend. Wie abgekoppelt lässt sich der Calculator wirklich realisieren?
			\\ & &
			Vorstellung der Hauptbibliothek die für die (aufwändigen) Rechnungen wie Nullstellenberechnung benutzt werden soll.
			\\ & &
			Warum Apache Commons Math und nicht JScience?
			\\ & &
			Diskussion des Programm-Workflows.
	\\ \hline
		\textbf{05.01.2020} 
			& \textbf{\hfill120} & \textbf{Gruppenmeeting 6 - Abstimmung der Teams}
			\\ & &
			\small{\textit{Teilnehmer: Bockhorn, Falk, Gentges, Istogu, Keienburg, Meinerzhagen, Pham, Schwenke}}
			\\ & &
			Aufnahme des aktuellen Projektstands.
			\\ & &
			Besprechung des weiteren Vorgehens.
			\\ & &
			Abstimmung de Aufgaben.
			\\ & &
			Besprechung des geplanten Frontends.
			\\ & &
			Besprechung / Lösung von Problemen.
		\\ \cline{2-3}
		& \textbf{\hfill180} & \textbf{GUI-Team-Meeting 5 - Abstimmung der Teams} 
			\\ & &
			\small{\textit{Teilnehmer: Bockhorn, Gentges, Pham}}
			\\ & &
			Besprechung des Frontends und Einarbeitung des Feedbacks
			\\ & &
			Recherche nach Emulator Skins
			\\ & &
			Erste Drafts des Layouts
			\\ & &
			Anlegen der Kacheln (Style resource)	
	\\ \hline
		\textbf{10.01.2020} 
			& \textbf{\hfill90} & \textbf{GUI-Team-Meeting 6 - Layouts}
			\\ & &
			\small{\textit{Teilnehmer: Bockhorn, Gentges, Pham}}
			\\ & &
			Weitere Standardlayouts finalisiert
			\\ & &
			Kachelarchitektur für Frontend finalisiert			
	\\ \hline
		\textbf{14.01.2020} 
			& \textbf{\hfill90} & \textbf{Gruppenmeeting 7 - Integration der Komponenten}
			\\ & &
			\small{\textit{Teilnehmer: Bockhorn, Falk, Gentges, Istogu, Keienburg, Meinerzhagen, Pham, Schwenke}}
			\\ & &
			Zusammenführung Frontend Backend
			\\ & &
			Präsentation des Frontends durch das GUI-Team.
			\\ & &
			Besprechen von MVP-Umsetzung in Android.
			\\ & &
			Backend Unit-Testing Fortschritte.
			\\ & &
			Serialisierung der Stacks zur Session-Sicherung.
	\\ \hline
		\textbf{24.01.2020} & 
			\textbf{\hfill240} & \textbf{Architektur- und Backend-Team-Meeting}
			\\ & &
			\small{\textit{Teilnehmer: Falk, Istogu, Keienburg, Meinerzhagen, Schwenke}}
			\\ & &
			Detaillierte Ausarbeitung der Architektur im Backend.
			\\ & &
			Programmieren im Team inklusive der Zusammenführung der Features
			\\ & &
			Umbau der Programmstruktur zur Anpassung an MVP
	\\ \hline
		\textbf{28.01.2020} 
			& \textbf{\hfill90} & \textbf{Prototyp Vorstellung}
			\\ & &
			\small{\textit{Teilnehmer: Bockhorn, Falk, Istogu, Meinerzhagen, Schwenke Herr Prof. Dr. Seifert}}
			\\ & &
			Gespräch mit Herr Prof. Dr. Thomas Seifert über den aktuellen Stand des Projekts und im Anschluss daran eine Nachbesprechung innerhalb des Teams.
			\\ & &
			Vorstellung:
			\begin{itemize}\renewcommand\labelitemi{--}
				\item Vorstellung der bereits implementierten Grundfunktionen der App.
				\item Vorstellung des verwendeten Design-Patterns.
				\item Abgleich von Umsetzung mit den Anforderungen des Dozenten.
				\item Ansatz des Backends erklärt.
				\item Gerät ausleihen, um nicht nur mit Emulator testen zu können.
				\item Serialisierung der Daten (Speichern und Laden).
			\end{itemize}
			\\ & &
			Ergebnis:
			\begin{itemize}\renewcommand\labelitemi{--}
				\item Projekt ist auf einem guten Weg. Priorisiert werden sollen differenzierende Funktionen anstatt wenige Features sehr detailliert auszuarbeiten (Prototypische Arbeit).
				\item Ternäre, Quaternäre usw. Operationen sind gewünscht.
				\item Vektoren in Bestandteile lösen.
				\item Eingabe von Matrizen.
				\item Jeder Klasse muss ein Verantwortlicher zugeordnet sein.
			\end{itemize}		
		\\ \cline{2-3}
		& \textbf{\hfill90} & \textbf{Gruppenmeeting 8 - Nachbesprechung Vorstellung}
			\\ & &
			\small{\textit{Teilnehmer: Bockhorn, Falk, Gentges, Istogu, Keienburg, Meinerzhagen, Pham, Schwenke}}
			\\ & &
			Ergebnisse der Nachbesprechung
			\begin{itemize}\renewcommand\labelitemi{--}
				\item ''Vektor bauen'' / ''Vektoren auflösen'' Action. 
				\item Summe von Stack Action.
				\item 1x Triple Operator einfügen.
				\item Operanden Eingabe via einzelne Menüs.
				\item Ranks der Stacks anpassen.
				\item Format des ersten Stacks anpassen.
			\end{itemize}
		\\ \cline{2-3}
		& \textbf{\hfill120} & \textbf{GUI-Team-Meeting 7 - Tests \& Menüs } 
			\\ & &
			\small{\textit{Teilnehmer: Bockhorn, Gentges, Pham}}
			\\ & &
			Funktionstests des Frontends 
			\\ & &
			Beginn der Implementierung der Menüführung			
	\\ \hline
		\textbf{03.02.2020} 
			& \textbf{\hfill90} & \textbf{Gruppenmeeting 9 - Absprachen \& Dokumentation}
			\\ & &
			\small{\textit{Teilnehmer: Bockhorn, Falk, Gentges, Istogu, Keienburg, Meinerzhagen, Pham, Schwenke}}
			\\ & &
			Besprechen des aktuellen Stands der App.
			\\ & &
			Was muss noch unbedingt umgesetzt werden?
			\\ & &
			Aufteilung der noch offenen Kapitel in der Ausarbeitung.
			\\ & &
			Neues Kapitel ''Einleitung'' mit Motivation.
			\\ & &
			Anpassung einiger Kapitelbezeichnungen an Gegebenheiten des Projekts.
			\\ & &
			Koordination der Ausarbeitung.
		\\ \cline{2-3}
		& \textbf{\hfill50} & \textbf{Backend-Team-Meeting 2 - Finalisierung}
			\\ & &
			\small{\textit{Teilnehmer: Istogu, Keienburg}}
			\\ & &
			Absprache der letzten zu anzupassenden Funktionalitäten
	\\ \hline
		\textbf{05.02.2020} 
			& \textbf{\hfill90} & \textbf{Gruppenmeeting 10 - Finalisierung \& Dokumentation}
			\\ & &
			\small{\textit{Teilnehmer: Bockhorn, Falk, Gentges, Istogu, Keienburg, Meinerzhagen, Pham, Schwenke}}
			\\ & &
			Ergebnisbesprechung für das App-Debugging.
			\\ & &
			Gemeinsames Arbeiten an der Dokumentation.
	\\ \hline\hline
\end{longtable}
}

\clearpage

\subsection{Projekttagebücher}

\subsubsection{Tom Bockhorn}

{\def\arraystretch{1.25}\tabcolsep=5pt
	\begin{longtable}{|l|l|p{11cm}|}
		\hline
		\textbf{Datum} & \textbf{Dauer} & \textbf{Beschreibung}
		\\ \hline \hline
		\endfirsthead
		\hline
		\endhead
		\hline
		\endfoot
		\multicolumn{3}{|c|}{\textit{Summe der Dauer aller Aktivitäten: 3.840 Minuten}}
		\\ \hline
		\endlastfoot
		
		\textbf{03.09.2019} 
			& \textbf{\hfill 200} & Gruppenmeeting 1 \\\cline{2-3}
			& \textbf{\hfill 120} & Recherche zur programmatischen Umsetzung von Taschenrechnern \\\cline{2-3}
			& \textbf{\hfill 30} & Einrichten von Android Studio mit Emulator
		\\	
		\hline \textbf{04.09.2019}
			& \textbf{\hfill 110} & Gruppenmeeting 2 \\\cline{2-3}
			& \textbf{\hfill 90} & Auf Basis des Feedbacks wurde das Konzept-Design überarbeitet und angepasst
		\\	
		\hline \textbf{05.09.2019}
			& \textbf{\hfill 60} & GUI-Team-Meeting 1 \\\cline{2-3}
			& \textbf{\hfill 60} & Gruppenmeeting 3 \\
			& \textbf{\hfill 80} & Konzeptionierung der Bearbeitungsfunktionalitäten des Taschenrechners. \\
			& & Erstellung und Bearbeitung des Mid-Fidelity-Prototypens \\
			& & Besprechung des aktuellen Standes
		\\	
		\hline \textbf{13.09.2019}
			& \textbf{\hfill 120} & GUI-Team-Meeting 2
		\\	
		\hline \textbf{17.09.2019}
			& \textbf{\hfill 90} & GUI-Team-Meeting 3
		\\	
		\hline \textbf{23.09.2019}
			& \textbf{\hfill 120} & GUI-Team-Meeting 4
		\\	
		\hline \textbf{26.09.2019}
			& \textbf{\hfill 150} & Gruppenmeeting 4
		\\			
		\hline \textbf{09.10.2019}
			& \textbf{\hfill 90} & Gruppenmeeting 5
		\\	
		\hline \textbf{05.01.2020}
			& \textbf{\hfill 120} & Gruppenmeeting 6 \\\cline{2-3}
			& \textbf{\hfill 180} & GUI-Team-Meeting 5
		\\	
		\hline \textbf{06.01.2020}
			& \textbf{\hfill 90} & Anlesen von Activity Verhalten in Android \\\cline{2-3}
			& \textbf{\hfill 360} & Konzeption der Architektur für generische Kacheln im Frontend \\
			& & Konzeption eines Containers für Kacheln im Frontend
		\\	
		\hline \textbf{07.01.2020}
			& \textbf{\hfill 180} & Aufteilung der Kacheln in kontextnah und kontextfern
		\\	
		\hline \textbf{10.01.2020}
			& \textbf{\hfill 120} & Entwicklung der Container mit aufgeteilter Kachelarchitektur \\
			& & Finalisierung der Container und Testen \\\cline{2-3}	
			& \textbf{\hfill 90} & GUI-Team-Meeting 6
		\\	
		\hline \textbf{11.01.2020}
			& \textbf{\hfill 90} & Fehlerbehebung in den Containern \\
			& & Schulung anderer Teammitglieder über die generische Kachelarchitektur
		\\	
		\hline \textbf{14.01.2020}
			& \textbf{\hfill 90} & Gruppenmeeting 7 \\\cline{2-3}
			& \textbf{\hfill 120} & Testen der Container Implementierung und Optimierung der Lade und Speichermodule 
		\\	
		\hline \textbf{28.01.2020}
			& \textbf{\hfill 90} & Prototyp Vorstellung \\\cline{2-3}
			& \textbf{\hfill 90} & Gruppenmeeting 8 \\
			& \textbf{\hfill 120} & GUI-Team-Meeting 7
		\\	
		\hline \textbf{02.02.2020}
			& \textbf{\hfill 120} & Einlesen in mögliche Menüführungen in Android
		\\	
		\hline \textbf{03.02.2020}
			& \textbf{\hfill 120} & Konzeption und Entwicklung der Rangauswahlmenüs für Stack und Historie \\\cline{2-3}
			& \textbf{\hfill 60} & Integration der Rangauswahlmenü in die restlichen Menüs \\
			& \textbf{\hfill 90} & Gruppenmeeting 9
		\\	
		\hline \textbf{04.02.2020}
			& \textbf{\hfill 300} & Fehler der Layout Container beheben \\
			& & Unterstützende Fehlerbehebende Arbeit in anderen Bereichen der Software
		\\	
		\hline \textbf{05.02.2020}
			& \textbf{\hfill 90} & Gruppenmeeting 10
		\\ \hline\hline
	\end{longtable}
}

\clearpage

\subsubsection{Hendrik Falk}

{\def\arraystretch{1.25}\tabcolsep=5pt
	\begin{longtable}{|l|l|p{11cm}|}
		\hline
		\textbf{Datum} & \textbf{Dauer} & \textbf{Beschreibung}
		\\ \hline \hline
		\endfirsthead
		\hline
		\endhead
		\hline
		\endfoot
		\multicolumn{3}{|c|}{\textit{Summe der Dauer aller Aktivitäten: x Minuten}}
		\\ \hline
		\endlastfoot
		
		\textbf{03.09.2019} 
			& \textbf{\hfill 200} & Gruppenmeeting 1 \\\cline{2-3}
			& \textbf{\hfill x} & x \\\cline{2-3}
			& \textbf{\hfill x} & x 
		\\ \hline \textbf{04.09.2019}
			& \textbf{\hfill x} & x \\\cline{2-3}
			& \textbf{\hfill x} & x
		\\ \hline \textbf{05.09.2019}
			& \textbf{\hfill x} & x \\\cline{2-3}
			& \textbf{\hfill x} & x
		\\ \hline \textbf{17.09.2019}
			& \textbf{\hfill x} & x \\\cline{2-3}
			& \textbf{\hfill x} & x
		\\ \hline \textbf{26.09.2019}
			& \textbf{\hfill x} & x \\\cline{2-3}
			& \textbf{\hfill x} & x
		\\ \hline \textbf{09.10.2019}
			& \textbf{\hfill x} & x \\\cline{2-3}
			& \textbf{\hfill x} & x
		\\ \hline \textbf{20.12.2019}
			& \textbf{\hfill x} & x \\\cline{2-3}
			& \textbf{\hfill x} & x
		\\ \hline \textbf{02.01.2020}
			& \textbf{\hfill x} &x\\\cline{2-3}
			& \textbf{\hfill x} & x
		\\ \hline \textbf{03.01.2020}
			& \textbf{\hfill x} & x\\\cline{2-3}
			& \textbf{\hfill x} & x
		\\ \hline \textbf{05.01.2020}
			& \textbf{\hfill x} & x \\\cline{2-3}
			& \textbf{\hfill x} & x
		\\ \hline \textbf{06.01.2020}
			& \textbf{\hfill x} &x \\\cline{2-3}
			& \textbf{\hfill x} & x
		\\ \hline \textbf{07.01.2020}
			& \textbf{\hfill x} & x \\\cline{2-3}	
			& \textbf{\hfill x} & x
		\\ \hline \textbf{08.01.2020}
			& \textbf{\hfill x} & x \\\cline{2-3}
			& \textbf{\hfill x} & x
		\\ \hline \textbf{14.01.2020}
			& \textbf{\hfill x} & x \\\cline{2-3}
			& \textbf{\hfill x} & x
		\\ \hline \textbf{24.01.2020}
			& \textbf{\hfill x} & x \\\cline{2-3}
			& \textbf{\hfill x} & x
		\\ \hline \textbf{27.01.2020}
			& \textbf{\hfill x} & x \\\cline{2-3}
			& \textbf{\hfill x} & x
		\\ \hline \textbf{28.01.2020}
			& \textbf{\hfill x} & x \\\cline{2-3}
			& \textbf{\hfill x} & x
		\\ \hline \textbf{30.01.2020}
			& \textbf{\hfill x} & x \\\cline{2-3}
			& \textbf{\hfill x} & x
		\\ \hline \textbf{03.02.2020}
			& \textbf{\hfill x} & x \\\cline{2-3}
			& \textbf{\hfill x} & x
		\\ \hline \textbf{04.02.2020}
			& \textbf{\hfill x} & x \\\cline{2-3}
			& \textbf{\hfill x} & x
		\\ \hline \textbf{05.02.2020}
			& \textbf{\hfill x} & x \\\cline{2-3}
			& \textbf{\hfill x} & x	\\
		\hline\hline
	\end{longtable}
}

\clearpage

\subsubsection{Dennis Gentges}

{\def\arraystretch{1.25}\tabcolsep=5pt
	\begin{longtable}{|l|l|p{11cm}|}
		\hline
		\textbf{Datum} & \textbf{Dauer} & \textbf{Beschreibung}
		\\ \hline \hline
		\endfirsthead
		\hline
		\endhead
		\hline
		\endfoot
		\multicolumn{3}{|c|}{\textit{Summe der Dauer aller Aktivitäten: x Minuten}}
		\\ \hline
		\endlastfoot
		
		\textbf{03.09.2019} 
			& \textbf{\hfill 200} & Gruppenmeeting 1 \\\cline{2-3}
			& \textbf{\hfill x} & x \\\cline{2-3}
			& \textbf{\hfill x} & x 
		\\ \hline \textbf{04.09.2019}
			& \textbf{\hfill x} & x \\\cline{2-3}
			& \textbf{\hfill x} & x
		\\ \hline \textbf{05.09.2019}
			& \textbf{\hfill x} & x \\\cline{2-3}
			& \textbf{\hfill x} & x
		\\ \hline \textbf{08.09.2019}
			& \textbf{\hfill x} & x \\\cline{2-3}
			& \textbf{\hfill x} & x
		\\ \hline \textbf{10.09.2019}
			& \textbf{\hfill x} & x \\\cline{2-3}
			& \textbf{\hfill x} & x
		\\ \hline \textbf{13.10.2019}
			& \textbf{\hfill x} & x \\\cline{2-3}
			& \textbf{\hfill x} & x
		\\ \hline \textbf{17.09.2019}
			& \textbf{\hfill x} & x \\\cline{2-3}
			& \textbf{\hfill x} & x
		\\ \hline \textbf{23.09.2019}
			& \textbf{\hfill x} &x\\\cline{2-3}
			& \textbf{\hfill x} & x
		\\ \hline \textbf{26.09.2019}
			& \textbf{\hfill x} & x\\\cline{2-3}
			& \textbf{\hfill x} & x
		\\ \hline \textbf{01.10.2019}
			& \textbf{\hfill x} & x \\\cline{2-3}
			& \textbf{\hfill x} & x
		\\ \hline \textbf{09.10.2019}
			& \textbf{\hfill x} &x \\\cline{2-3}
			& \textbf{\hfill x} & x
		\\ \hline \textbf{16.11.2019}
			& \textbf{\hfill x} & x \\\cline{2-3}	
			& \textbf{\hfill x} & x
		\\ \hline \textbf{05.01.2020}
			& \textbf{\hfill x} & x \\\cline{2-3}
			& \textbf{\hfill x} & x
		\\ \hline \textbf{12.01.2020}
			& \textbf{\hfill x} & x \\\cline{2-3}
			& \textbf{\hfill x} & x
		\\ \hline \textbf{14.01.2020}
			& \textbf{\hfill x} & x \\\cline{2-3}
			& \textbf{\hfill x} & x
		\\ \hline \textbf{24.01.2020}
			& \textbf{\hfill x} & x \\\cline{2-3}
			& \textbf{\hfill x} & x
		\\ \hline \textbf{28.01.2020}
			& \textbf{\hfill x} & x \\\cline{2-3}
			& \textbf{\hfill x} & x
		\\ \hline \textbf{31.01.2020}
			& \textbf{\hfill x} & x \\\cline{2-3}
			& \textbf{\hfill x} & x
		\\ \hline \textbf{02.02.2020}
			& \textbf{\hfill x} & x \\\cline{2-3}
			& \textbf{\hfill x} & x
		\\ \hline \textbf{03.02.2020}
			& \textbf{\hfill x} & x \\\cline{2-3}
			& \textbf{\hfill x} & x
		\\ \hline \textbf{05.02.2020}
			& \textbf{\hfill x} & x \\\cline{2-3}
			& \textbf{\hfill x} & x	\\
		\hline\hline
	\end{longtable}
}

\clearpage

\subsubsection{Getuart Istogu}
{\def\arraystretch{1.25}\tabcolsep=5pt
	\begin{longtable}{|l|l|p{11cm}|}
		\hline
		\textbf{Datum} & \textbf{Dauer} & \textbf{Beschreibung}
		\\ \hline \hline
		\endfirsthead
		
		\hline
		\endhead
		
		\hline
		\endfoot
		
		\multicolumn{3}{|c|}{\textit{Summe der Dauer aller Aktivitäten: 4.370 Minuten}}
		\\ \hline
		\endlastfoot
		
		\textbf{03.09.2019} 
		& \textbf{\hfill200} & Gruppenmeeting 1
		\\ \hline
		
		\textbf{04.09.2019} 
		& \textbf{\hfill110} & Gruppenmeeting 2
		\\ \hline
		
		\textbf{04.09.2019} 
		& \textbf{\hfill40} & Ideen für mögliche Operatoren und Operanden sammeln
		\\ \hline
		
		\textbf{04.09.2019} 
		& \textbf{\hfill10} & Offizielle und interne Deadlines in Planner eintragen
		\\ \hline
		
		\textbf{05.09.2019} 
		& \textbf{\hfill60} & Gruppenmeeting 3
		\\ \hline
		
		\textbf{05.09.2019} 
		& \textbf{\hfill60} & Backend-Team-Meeting 1
		\\ \hline
		
		\textbf{17.09.2019} 
		& \textbf{\hfill60} & Architektur-Team-Meeting 2
		\\ \hline
		
		\textbf{26.09.2019} 
		& \textbf{\hfill150} & Gruppenmeeting 4
		\\ \hline
		
		\textbf{09.10.2019} 
		& \textbf{\hfill90} & Gruppenmeeting 5
		\\ \hline
		
		
		\textbf{12.10.2019} 
		& \textbf{\hfill30} & 
				Ermitteln der benötigten Operanden-Klassen (z.B. Matrix etc.) \\ & &
				
				Ermitteln der benötigten Operatoren-Klassen (Addition, Multiplikaiton usw.)
		\\ \hline
		
		\textbf{26.10.2019} 
		& \textbf{\hfill30} & 
				Recherche auf JScience und Apache Commons Math \\ & &
				Vergleichen der Ergebnisse in der Gruppe \\ & &
				Bereitstellung der Klassen/des Quellcodes \\ & &
				Überlegungen wie diese zu importieren sind 
		\\ \hline
		
		\textbf{09.11.2019} 
		& \textbf{\hfill80} & Durchführung des Cognitive-Walkthroughs
		\\ \hline
		
		\textbf{04.01.2020} 
		& \textbf{\hfill120} & Gruppenmeeting 6
		\\ \hline
		
		\textbf{04.01.2020} 
		& \textbf{\hfill120} & 
				Auseinandersetzung des bereits implementierten Code von den anderen Teammitglieder \\ & &
				Implementierung des Backends (Potenz, Wurzel, Modulo) \\ & &
				Auseinandersetzung über mögliche Gestaltung der Exception (+ Diskussion mit ein paar Teammitglieder)
		\\ \hline
		
		\textbf{14.01.2020} 
		& \textbf{\hfill90} & Gruppenmeeting 7
		\\ \hline
		
		\textbf{19.01.2020} 
		& \textbf{\hfill200} & 
			Klasse: Integral implementieren und zugehörige UnitTest-Klasse geschrieben \\ & &
			Bietet Apache Commons Math Integralrechnung an? \\ & &
			Umsetzung der Funktion für das Integral \\ & &
			Eigene Implementation für das Bilden von Stammfunktionen \\  & &
			Testklasse geschrieben
		\\ \hline
		
		\textbf{19.01.2020} 
		& \textbf{\hfill90} & Erstellung von Testklassen für die Klassen Potenz, Wurzel, Modulo
		\\ \hline
		
		\textbf{22.01.2020} 
		& \textbf{\hfill350} & Klasse: Limes und zugehörige UnitTest-Klasse geschrieben
		\\ \hline
		
		\textbf{24.01.2020} 
		& \textbf{\hfill240} & Architektur- und Backend-Team-Meeting
		\\ \hline
		
		\textbf{28.01.2020} 
		& \textbf{\hfill400} & Klasse: UtilMatrix und zugehörige UnitTest-Klasse implementiert
		\\ \hline
		
		\textbf{28.01.2020} 
		& \textbf{\hfill90} & Gruppenmeeting 8
		\\ \hline
		
		\textbf{29.01.2020} 
		& \textbf{\hfill120} & Verständnis für die Frontend-Klassen und deren Interaktionen gewinnen
		\\ \hline
		
		\textbf{29.01.2020} 
		& \textbf{\hfill150} & Überlegung der Architektur für die Menügestaltung
		\\ \hline
		
		\textbf{29.01.2020} 
		& \textbf{\hfill270} & Vorherige Implementierung abstrahieren 
			Implementierung der abstrakten Klasse DialogMenu
		\\ \hline
		
		\textbf{30.01.2020} 
		& \textbf{\hfill300} & Implementierung der Klasse InputTileType und Einbinden mit ClickListener
		\\ \hline
		
		\textbf{03.02.2020} 
		& \textbf{\hfill90} & Gruppenmeeting 9
		\\ \hline
		
		\textbf{04.02.2020} 
		& \textbf{\hfill420} & Dokumentation des Projektes
		\\ \hline
		
		\textbf{05.02.2020} 
		& \textbf{\hfill90} & Gruppenmeeting 10
		\\ \hline
		
		\textbf{05.02.2020} 
		& \textbf{\hfill30} & Projekttagebuch pflegen
		\\ \hline
		
		\textbf{05.02.2020} 
		& \textbf{\hfill280} & Dokumentation des Projekts (Teil 2)	
		\\ \hline\hline
		
	\end{longtable}
}

\clearpage

\subsubsection{Jannis Luca Keienburg}

{\def\arraystretch{1.25}\tabcolsep=5pt
	\begin{longtable}{|l|l|p{11cm}|}
		\hline
		\textbf{Datum} & \textbf{Dauer} & \textbf{Beschreibung}
		\\ \hline \hline
		\endfirsthead
		\hline
		\endhead
		\hline
		\endfoot
		\multicolumn{3}{|c|}{\textit{Summe der Dauer aller Aktivitäten: x Minuten}}
		\\ \hline
		\endlastfoot
		
		\textbf{03.09.2019} 
			& \textbf{\hfill 200} & Gruppenmeeting 1 
		\\ \hline \textbf{04.09.2019}
			& \textbf{\hfill 30} & Erstellung eines Entwurfs \\\cline{2-3}
			& \textbf{\hfill 110} & Gruppenmeeting 2 \\\cline{2-3}
			& \textbf{\hfill 60} & Nachbearbeitung des Entwurfs 
		\\ \hline \textbf{05.09.2019}
			& \textbf{\hfill 60} & Einrichtung von Android Studio \\\cline{2-3}
			& \textbf{\hfill 60} & Gruppenmeeting 3 \\\cline{2-3}
			& \textbf{\hfill 40} & Backend-Team-Meeting \\\cline{2-3}
			& \textbf{\hfill 20} & Zuweisung von Arbeitspaketen
		\\ \hline \textbf{17.09.2019}
			& \textbf{\hfill 150} & Architekturmeeting 2, Überlegungen zur Backend Architektur
		\\ \hline \textbf{09.10.2019}
			& \textbf{\hfill 90} & Gruppenmeeting 5
		\\ \hline \textbf{12.10.2019}
			& \textbf{\hfill 30} & Ermittlung der Benötigten Klassen, Ermittlung der benötigten Methoden für obige Klassen
		\\ \hline \textbf{26.10.2019}
			& \textbf{\hfill 30} & Recherche auf JScience und Apache Commons Math\\Vergleichen der Ergebnisse in der Gruppe\\Bereitstellung der Klassen / des Quellcodes\\Überlegungen, wie diese zu implementieren ist
		\\ \hline \textbf{01.01.2020}
			& \textbf{\hfill 200} & Konzeption der Stackoperatoren
		\\ \hline \textbf{02.01.2020}
			& \textbf{\hfill 250} & Überarbeiten des Konzeptes, erste Implementierungsversuche 
		\\ \hline \textbf{03.01.2020}
			& \textbf{\hfill 300} & Finalisieren des Konzeptes\\Fertigstellung der Implementierung der Stackoperationen 
		\\ \hline \textbf{04.01.2020}
			& \textbf{\hfill 120} & Gruppenmeeting 7 \\\cline{2-3}
			& \textbf{\hfill 120} & Verschaffung eines Überblicks über den aktuellen Quellcode\\Programmierung der Funktionen für Sinus, Cosinus, Tangens
		\\ \hline \textbf{11.01.2020}
			& \textbf{\hfill 120} & Implementierung des natürlichen Logarithmus\\Logarithmus zur Basis 10\\Logarithmus für beliebige Basis
		\\ \hline \textbf{14.01.2020}
			& \textbf{\hfill 45} & Implementierung der Ableitung 
		\\ \hline \textbf{15.01.2020}
			& \textbf{\hfill 60} & Implementierung der Nullstellenberechnung 
		\\ \hline \textbf{16.01.2020}
			& \textbf{\hfill 100} & Arbeit an der Funktion für Berechnung von Hoch- und Tiefpunkten
		\\ \hline \textbf{17.01.2020}
			& \textbf{\hfill 30} & Korrekturen an der Klasse für Hoch- und Tiefpunkte
		\\ \hline \textbf{22.01.2020}
			& \textbf{\hfill 120} & Erstellung von Unit Tests für Sinus, Cosinus, Tangens, Arcsinus, Arccosinus, Arctangens, Logarithmusfunktionen
		\\ \hline \textbf{23.01.2020}
			& \textbf{\hfill 150} & Erstellung von Unit Tests für Nullstellenberechnung, Ableitung, Hoch- und Tiefpunktberechnung
		\\ \hline \textbf{28.01.2020}
			& \textbf{\hfill 90} & Gruppenmeeting 8 \\\cline{2-3}
			& \textbf{\hfill 90} & Ausarbeitung, wie das LaTeXdokument erstellt werden soll \\\cline{2-3}
			& \textbf{\hfill 60} & Auswahl und Test der gewählten Software für LaTeX
		\\ \hline \textbf{31.01.2020}
			& \textbf{\hfill 150} & Dokumentation der Klassen: Derivation, zeros, HighAndLowPoints
		\\ \hline \textbf{01.02.2020}
			& \textbf{\hfill 120} & Dokumentation der Stackoperationen 
		\\ \hline \textbf{02.02.2020}
			& \textbf{\hfill x} & Dokumentation der Klassen:\\Sinus, Arcsinus, Cosunis, Arccosinus,\\Tangens, Arctangens, Logarithm, Logarithm10
		\\ \hline \textbf{03.02.2020}
			& \textbf{\hfill 90} & Gruppenmeeting 9
		\\ \hline \textbf{04.02.2020}
			& \textbf{\hfill 60} & Kontrolllesen der eigenen Ausarbeitung 
		\\ \hline \textbf{05.02.2020}
			& \textbf{\hfill 150} & Kontrolle der gesamten Ausarbeitung \\\cline{2-3}
			& \textbf{\hfill 90} & Gruppenmeeting 10 \\\cline{2-3}
			& \textbf{\hfill 30} & Projekttagebuch \\
		\hline\hline
	\end{longtable}
}

\clearpage

\subsubsection{Tim Jonas Meinerzhagen}

{\def\arraystretch{1.25}\tabcolsep=5pt
	\begin{longtable}{|l|l|p{11cm}|}
		\hline
		\textbf{Datum} & \textbf{Dauer} & \textbf{Beschreibung}
		\\ \hline \hline
		\endfirsthead
		\hline
		\endhead
		\hline
		\endfoot
		\multicolumn{3}{|c|}{\textit{Summe der Dauer aller Aktivitäten: x Minuten}}
		\\ \hline
		\endlastfoot
		
		\textbf{03.09.2019} 
			& \textbf{\hfill 200} & Gruppenmeeting 1 \\\cline{2-3}
			& \textbf{\hfill x} & x \\\cline{2-3}
			& \textbf{\hfill x} & x 
		\\ 
		\hline \textbf{04.09.2019}
			& \textbf{\hfill x} & x \\\cline{2-3}
			& \textbf{\hfill x} & x
		\\ 
		\hline \textbf{05.09.2019}
			& \textbf{\hfill x} & x \\\cline{2-3}
			& \textbf{\hfill x} & x \\
			& \textbf{\hfill x} & x \\
			& \textbf{\hfill x} & x \\
			& \textbf{\hfill x} & x
		\\ 
		\hline \textbf{17.09.2019}
			& \textbf{\hfill x} & x \\\cline{2-3}
			& \textbf{\hfill x} & x
		\\ 
		\hline \textbf{24.09.2019}
			& \textbf{\hfill x} & x
		\\ 
		\hline \textbf{26.09.2019}
			& \textbf{\hfill x} & x \\\cline{2-3}
			& \textbf{\hfill x} & x
		\\ 
		\hline \textbf{09.10.2019}
			& \textbf{\hfill x} & x \\\cline{2-3}
			& \textbf{\hfill x} & x
		\\ 
		\hline \textbf{05.11.2019}
			& \textbf{\hfill x} &x\\\cline{2-3}
			& \textbf{\hfill x} & x
		\\ 
		\hline \textbf{05.01.2019}
			& \textbf{\hfill x} & x\\\cline{2-3}
			& \textbf{\hfill x} & x
		\\ 
		\hline \textbf{07.01.2019}
			& \textbf{\hfill x} & x \\\cline{2-3}
			& \textbf{\hfill x} & x
		\\ 
		\hline \textbf{12.11.2019}
			& \textbf{\hfill x} &x \\\cline{2-3}
			& \textbf{\hfill x} & x
		\\ 
		\hline \textbf{13.11.2019}
			& \textbf{\hfill x} & x \\\cline{2-3}	
			& \textbf{\hfill x} & x
		\\ 
		\hline \textbf{08.01.2020}
			& \textbf{\hfill x} & x \\\cline{2-3}
			& \textbf{\hfill x} & x
		\\ 
		\hline \textbf{14.01.2020}
			& \textbf{\hfill x} & x \\\cline{2-3}
			& \textbf{\hfill x} & x
		\\ 
		\hline \textbf{24.01.2020}
			& \textbf{\hfill x} & x \\\cline{2-3}
			& \textbf{\hfill x} & x
		\\ 
		\hline \textbf{26.01.2020}
			& \textbf{\hfill x} & x \\\cline{2-3}
			& \textbf{\hfill x} & x
		\\ 
		\hline \textbf{27.01.2020}
			& \textbf{\hfill x} & x \\\cline{2-3}
			& \textbf{\hfill x} & x
		\\ 
		\hline \textbf{28.01.2020}
			& \textbf{\hfill x} & x \\\cline{2-3}
			& \textbf{\hfill x} & x
		\\ 
		\hline \textbf{02.02.2020}
			& \textbf{\hfill x} & x \\\cline{2-3}
			& \textbf{\hfill x} & x
		\\ 
		\hline \textbf{04.02.2020}
			& \textbf{\hfill x} & x \\\cline{2-3}
			& \textbf{\hfill x} & x
		\\ 
		\hline \textbf{05.02.2020}
			& \textbf{\hfill x} & x \\\cline{2-3}
			& \textbf{\hfill x} & x \\
		\hline\hline
	\end{longtable}
}

\clearpage

\subsubsection{Khang Pham}

{\def\arraystretch{1.25}\tabcolsep=5pt
	\begin{longtable}{|l|l|p{11cm}|}
		\hline
		\textbf{Datum} & \textbf{Dauer} & \textbf{Beschreibung}
		\\ \hline \hline
		\endfirsthead
		\hline
		\endhead
		\hline
		\endfoot
		\multicolumn{3}{|c|}{\textit{Summe der Dauer aller Aktivitäten: x Minuten}}
		\\ \hline
		\endlastfoot
		
		\textbf{03.09.2019} 
			& \textbf{\hfill 200} & Gruppenmeeting 1 \\\cline{2-3}
			& \textbf{\hfill x} & x \\\cline{2-3}
			& \textbf{\hfill x} & x 
		\\ \hline \textbf{04.09.2019}
			& \textbf{\hfill x} & x \\\cline{2-3}
			& \textbf{\hfill x} & x
		\\ \hline \textbf{05.09.2019}
			& \textbf{\hfill x} & x \\\cline{2-3}
			& \textbf{\hfill x} & x
		\\ \hline \textbf{13.09.2019}
			& \textbf{\hfill x} & x \\\cline{2-3}
			& \textbf{\hfill x} & x
		\\ \hline \textbf{17.09.2019}
			& \textbf{\hfill x} & x \\\cline{2-3}
			& \textbf{\hfill x} & x
		\\ \hline \textbf{23.09.2019}
			& \textbf{\hfill x} & x \\\cline{2-3}
			& \textbf{\hfill x} & x
		\\ \hline \textbf{26.09.2019}
			& \textbf{\hfill x} & x \\\cline{2-3}
			& \textbf{\hfill x} & x
		\\ \hline \textbf{09.10.2019}
			& \textbf{\hfill x} &x\\\cline{2-3}
			& \textbf{\hfill x} & x
		\\ \hline \textbf{05.01.2020}
			& \textbf{\hfill x} & x\\\cline{2-3}
			& \textbf{\hfill x} & x
		\\ \hline \textbf{07.01.2020}
			& \textbf{\hfill x} & x \\\cline{2-3}
			& \textbf{\hfill x} & x
		\\ \hline \textbf{10.01.2020}
			& \textbf{\hfill x} &x \\\cline{2-3}
			& \textbf{\hfill x} & x
		\\ \hline \textbf{14.01.2020}
			& \textbf{\hfill x} & x \\\cline{2-3}	
			& \textbf{\hfill x} & x
		\\ \hline \textbf{08.01.2020}
			& \textbf{\hfill x} & x \\\cline{2-3}
			& \textbf{\hfill x} & x
		\\ \hline \textbf{25.01.2020}
			& \textbf{\hfill x} & x \\\cline{2-3}
			& \textbf{\hfill x} & x
		\\ \hline \textbf{28.01.2020}
			& \textbf{\hfill x} & x \\\cline{2-3}
			& \textbf{\hfill x} & x
		\\ \hline \textbf{01.02.2020}
			& \textbf{\hfill x} & x \\\cline{2-3}
			& \textbf{\hfill x} & x
		\\ \hline \textbf{02.02.2020}
			& \textbf{\hfill x} & x \\\cline{2-3}
			& \textbf{\hfill x} & x
		\\ \hline \textbf{28.01.2020}
			& \textbf{\hfill x} & x \\\cline{2-3}
			& \textbf{\hfill x} & x
		\\ \hline \textbf{02.02.2020}
			& \textbf{\hfill x} & x \\\cline{2-3}
			& \textbf{\hfill x} & x
		\\ \hline \textbf{03.02.2020}
			& \textbf{\hfill x} & x \\\cline{2-3}
			& \textbf{\hfill x} & x
		\\ \hline \textbf{05.02.2020}
			& \textbf{\hfill x} & x \\\cline{2-3}
			& \textbf{\hfill x} & x \\
		\hline\hline
	\end{longtable}
}

\clearpage

\subsubsection{Tim Schwenke}

{\def\arraystretch{1.25}\tabcolsep=5pt
	\begin{longtable}{|l|l|p{11cm}|}
		\hline
		\textbf{Datum} & \textbf{Dauer} & \textbf{Beschreibung}
		\\ \hline \hline
		\endfirsthead
		\hline
		\endhead
		\hline
		\endfoot
		\multicolumn{3}{|c|}{\textit{Summe der Dauer aller Aktivitäten: 4.330 Minuten}}
		\\ \hline
		\endlastfoot
		
		\textbf{03.09.2019} 
			& \textbf{\hfill 200} & Gruppenmeeting 1 \\\cline{2-3}
			& \textbf{\hfill 120} & Recherche zur programmatischen Umsetzung von Taschenrechnern \\\cline{2-3}
			& \textbf{\hfill 30} & Einrichten von Android Studio mit Emulator \\\cline{2-3}
			& \textbf{\hfill 30} & Einrichten von zwei Git-Repositories für das Softwareprojekt selbst und die Dokumentation in Latex \\\cline{2-3}
			& \textbf{\hfill 30} & Erarbeitung Vorgehen Projekttagebücher 
		\\ \hline \textbf{04.09.2019}
			& \textbf{\hfill 110} & Gruppenmeeting 2 
		\\ \hline \textbf{05.09.2019}
			& \textbf{\hfill 60} & Architektur-Team-Meeting 1 \\\cline{2-3}
			& \textbf{\hfill 60} & Gruppenmeeting 3 \\\cline{2-3}
			& \textbf{\hfill 30} & Finalisierung und Abgabe Projekttagebuch
		\\ \hline \textbf{14.09.2019}
			& \textbf{\hfill 180} & Erstellung mehrerer Konzepte für die Kalkulationsorchestrierung 
		\\ \hline \textbf{15.09.2019}
			& \textbf{\hfill 180} & Einlesen in generische Programmierung mit Java
		\\ \hline \textbf{16.09.2019}
			& \textbf{\hfill 100} & Suchen und Ausprobieren von Mathebibliotheken für Java \\\cline{2-3}
			& \textbf{\hfill 60} & Genaueren Vergleich mithilfe von Beispielen zwischen Apache Commons Math und JScience durchführen und Entscheidung getroffen.
		\\ \hline \textbf{17.09.2019}
			& \textbf{\hfill 60} & Architektur-Team-Meeting 2
		\\ \hline \textbf{21.09.2019}
			& \textbf{\hfill 180} & Erstellung des Operanden Konzeptes
		\\ \hline \textbf{12.11.2019}
			& \textbf{\hfill 120} & Implementierung der Klasse Operand \\\cline{2-3}
			& \textbf{\hfill 60} & Implementierung von Hilfsklassen (\code{DoubleComparator} etc.)
		\\ \hline \textbf{26.09.2019}
			& \textbf{\hfill 150} & Gruppenmeeting 4
		\\ \hline \textbf{05.10.2019}
			& \textbf{\hfill 360} & Prototypische Entwicklung zweier Konzepte für Orchestrierung
		\\ \hline \textbf{06.10.2019}
			& \textbf{\hfill 360} & Umsetzung der finalen Schnittstelle für generische Kalkulationen
		\\ \hline \textbf{09.10.2019}
			& \textbf{\hfill 90} & Gruppenmeeting 5 \\\cline{2-3}
			& \textbf{\hfill 90} & Projektdokumentation
		\\ \hline \textbf{10.10.2019}
			& \textbf{\hfill 60} & Optimierung des Operand-Stacks
		\\ \hline \textbf{05.11.2019}
			& \textbf{\hfill 30} & Finalisierung und Abgabe Projekttagebuch
		\\ \hline \textbf{05.01.2020}
			& \textbf{\hfill 120} & Gruppenmeeting 6
		\\ \hline \textbf{08.01.2020}
			& \textbf{\hfill 30} & Finalisierung und Abgabe Projekttagebuch
		\\ \hline \textbf{14.01.2020}
			& \textbf{\hfill 90} & Gruppenmeeting 7
		\\ \hline \textbf{16.01.2020}
			& \textbf{\hfill 120} & Optimierung verschiedener Actions (inc. Debugging) \\\cline{2-3}
			& \textbf{\hfill 60} & Schreiben von Unit-Tests für einige Actions
		\\ \hline \textbf{24.01.2020}
			& \textbf{\hfill 240} & Architektur- und Backend-Meeting
		\\ \hline \textbf{28.01.2020}
			& \textbf{\hfill 90} & Prototyp Vorstellung \\\cline{2-3}
			& \textbf{\hfill 90} & Gruppenmeeting 8
		\\ \hline \textbf{01.02.2020}
			& \textbf{\hfill 120} & Projektdokumentation
		\\ \hline \textbf{03.02.2020}
			& \textbf{\hfill 90} & Gruppenmeeting 9 \\\cline{2-3}
			& \textbf{\hfill 50} & Backend-Team-Meeting 2
		\\ \hline \textbf{05.02.2020}
			& \textbf{\hfill 90} & Gruppenmeeting 10 \\\cline{2-3}
			& \textbf{\hfill 30} & Finalisierung und Abgabe Projekttagebuch \\\cline{2-3}
			& \textbf{\hfill 360} & Projektdokumentation \\
		\hline\hline
	\end{longtable}
}

\clearpage

\subsection{Beschreibung von Problemen}

Bei der Erstellung der Applikation kam es in den jeweiligen Teams zu verschiedenen Arten von Problemen, welche in den nachfolgenden Punkten näher erläutert werden.

\subsubsection{Programmierkenntnisse [Gentges]}

Aufgrund mangelnder vorhergegangener Erfahrungen in der Programmierung sowie mit GitHub und Latex fiel es mir zu Beginn schwer konkrete Programmieraufgaben zu übernehmen. Dadurch benötigte ich eine wesentlich größere Einarbeitungszeit als meine Kollegen was zu einem gewissen zeitlichen Verzug im Front End Team führte.

Nach der Einarbeitung versuchte ich mich an ersten Codierungsaufgaben, was jedoch insbesondere am Anfang der zusätzlichen Kontrolle meiner Kollegen bedurfte. 
Leider kam es hierbei bei der Codeimplementierung meinerseits zu Fehlern weshalb der Code nicht kompiliert werden konnte und zusätzlicher Aufwand bei der Fehlerfindung und Erklärung angefallen ist. 

\subsubsection{Vergleich von Doubles [Meinerzhagen]}
Beim Arbeiten mit Double Werten ist im Entwicklungsprozess immer wieder aufgefallen, dass Vergleiche zweier Doubles nicht das erwartete Ergebnis lieferten. Nach einiger Recherche konnte das Problem identifiziert werden. So wird mit der Standard Vergleichsmethode von Java sogar ein  Vergleich der zwei Doubles \code{1.1} und \code{1.1} ein \code{false} zurückgeben.

Dies ist durch das verbreitete IEE 754 Format für die Implementierung von Dezimalzahlen zu erklären. Dies hat zur Folge, dass Konversionen und Änderungen einer konkreten Dezimalzahl zu Rundungsfehlern führen. Das ist auch in Java der Fall.

Da innerhalb dieses Projekts alles auf Doubles basiert ist es notwendig für dieses Problem eine Lösung zu finden. So wurde die neue Klasse DoubleComparator erstellt. Alle Vergleiche müssen über diese Klasse erfolgen.

\subsubsection{Teamkommunikation in unterschiedlichen Umgebungen [Falk]}

Ein großes Problem für die Kommunikation innerhalb des Teams war die Situation, dass wir an verschiedenen Standorten arbeiteten. Die meisten von arbeiteten im Chempark Leverkusen, jedoch hatten wir auch Teammitglieder in Wuppertal und Monheim. Das erschwerte die Kommunikation, da vieles digital verlagert werden musste. Dadurch viel die sehr wichtige Komponente des Face-to-Face-Meetings sehr oft raus, jedoch nicht komplett, da wir uns oft auch an den Samstagen austauschen konnten, an welchen wir an Vorlesungsveranstaltungen teilnahmen. 

Bei der digitalen Kommunikation gab es allerdings auch verschiedene Probleme. Zum einen wurde einigen von uns keine Einrichtungen zum Telefonieren bereitgestellt, wodurch diese bei Meetings zwar zuhören konnten, aber selber nur über Text kommunizieren konnten. Auch waren die meisten von uns bei der Bayer AG angestellt, weshalb wir oft auch interne Kommunikationsmöglichkeiten nutzten, zu welchen die Teammitglieder, welche von der Currenta GmbH \& Co. OHG beschäftigt wurden, nicht immer Zugriff hatten. Das lag daran, dass in der Bearbeitungszeit Bayer seine Anteile an Currenta verkaufte und daher die vorher eng verbundenen IT-Systeme getrennt werden mussten.
