%!TEX root = ../Thesis.tex
\section{Ziel des Projektes [Falk]}

Das Ziel des Projektes ist die Entwicklung einer kachelbasierten Android-Applikation für den Auftraggeber Prof. Dr. Thomas Seifert. Diese Applikation soll die Funktion eines Taschenrechners nach der umgekehrten, polnischen Notation (UPN) erfüllen und wird im Folgenden als Tile Calculator bezeichnet. Dies stellt die Prüfungsleistung im Modul ''Projekte in der Wirtschaftsinformatik'' des Teams ''Das Proletariat'' dar. 

Das Team besteht aus folgenden Studierenden der Gruppe BFWI317B and der Fachhochschule der Wirtschaft Bergisch Gladbach: Tom Bockhorn, Hendrik Falk, Dennis Gentges, Getuart Istogu, Jannis Luca Keienburg, Tim Jonas Meinerzhagen, Khang Pham und Tim Schwenke. Diese absolvieren das Wirtschaftsinformatikstudium mit Schwerpunkt IT-Consulting als Angestellte und Auszubildende der Bayer AG, Bayer Business Services GmbH und der Currenta GmbH \& Co. OHG.

Der Projektzeitraum erstreckt sich vom 03.09.2019 bis zum 06.02.2020, wobei die Aufteilung der Zeit dem Team selbst überlassen wurde.

Die zu entwickelnde Applikation, sowie der Prozess zur Erstellung selbiger soll ausführlich dokumentiert werden und zusammen mit der Applikation, welche auf der zur Verfügung gestellten Hardware installiert sein muss, eingereicht werden. 

Die folgenden Termine müssen eingehalten werden, damit das Projekt als erfolgreich gilt: 

\begin{itemize}
	\item \textit{05.09.2019:} Hochladen der aktuellen Version des Projekttagebuchs und Vorlage beim Dozenten.
	\item \textit{05.11.2019:} Hochladen der aktuellen Version des Projekttagebuchs.
	\item \textit{08.01.2020:} Hochladen der aktuellen Version des Projekttagebuchs.
	\item \textit{05.02.2020:} Hochladen der aktuellen Version des Projekttagebuchs.
	\item \textit{06.02.2020:} Hochladen der Individualversion der Studienarbeit und des Projektes (Deadline: 17:00).
	\item \textit{08.02.2020:} Präsentation des Projektergebnisses und Abgabe der ausgedruckten Team-Version der Studienarbeit sowie des zur Verfügung gestellten Tablets.
\end{itemize}

Alle hochzuladenden Dateien werden im vom Auftraggeber erstellten Microsoft Teams abgegeben.