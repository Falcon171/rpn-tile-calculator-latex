%!TEX root = ../Thesis.tex

\section{Einleitung [Pham]}

''Viele Menschen organisieren ihr halbes Leben über Apps. Ohne Apps wüssten viele nicht mehr, wie das Wetter wird, wie der Kontostand lautet, wo der nächste Stau wartet oder wie sich ihr Lieblingsverein gerade schlägt.''\footnote{\cite[Berg, zitiert nach][]{rondinella2019}}

Apps gewinnen immer mehr an Bedeutung im Leben der Menschen. 2018 wurden in Deutschland zum ersten Mal mehr als zwei Milliarden Apps heruntergeladen. Damit wurde nicht nur ein neuer Rekord erreicht was die Anzahl an Downloads angeht, sondern auch die erwirtschafteten Umsätze erreichten einen neuen Höchstwert. Zwei Drittel der App-Downloads entfielen hierbei auf den Google Play Store. Auch für die kommenden Jahre wird ein Wachstum für den App-Markt vorausgesagt.\footnote{\cite[vgl.][]{rondinella2019}}

Aus diesem Grund beschäftigt sich die vorliegende Studienarbeit mit der Entwicklung einer App für Android. Die Entwicklung der App findet hierbei im Rahmen des FHDW Moduls ''Projekte der Wirtschaftsinformatik'' statt.

Das Modul Projekte der Wirtschaftsinformatik beinhaltet die Umsetzung eines komplexen Projektes inklusive der Präsentation und der Dokumentation der Projektergebnisse.\footnote{\cite[vgl.][]{fhdw2020}}  Die vom Team gewählte Aufgabenstellung war hierbei, einen kachelbasierten Taschenrechner nach der umgekehrten, polnischen Notation zu entwickeln. 

Dabei soll in der Studienarbeit nicht nur auf die Vorgehensweise der Teammitglieder während der Programmierung der App eingegangen werden. Vielmehr soll in dieser Studienarbeit auf alle Aspekte eingegangen werden, die für die Durchführung eines solchen Projektes benötigt werden. Dies beinhaltet organisatorische Aspekte wie Projekt- und Zeitplanung, die Koordination zwischen den Teammitgliedern und die Dokumentation des Projektes. Für die Durchführung des Projektes standen dem Team dabei etwas mehr als 80 Stunden pro Teammitglied über einen Zeitraum von 5 Monaten zur Verfügung, zusätzlich zu 40 Kontaktstunden in den Vorlesungen, in welchen jedoch hauptsächlich Grundlagenwissen in Python aufgebaut wurde, welches nicht relevant für die vom Team gewählte Aufgabenstellung war. 